	\begin{abstract}
		In today's ubiquitous computing environment, the demands to filter data from the noise and react to patterns of events expeditiously has intensified the burden put on the already complex event processing ecosystem. The complex event processing ecosystem is composed of several contributing technologies working in synergy to afford the needed processing for computation and communication as events progress through the system achieving higher levels of abstractions with producers and consumers at each level. The ecosystem can be viewed as an elaborate overlay on top of the underlying network. The thesis proposes a paradigm of viewing the underlying network as a contributing technology to the complex event processing ecosystem by offloading event processing application context onto the network. With the advent of software defined networking and its' consequent separation of the control and data planes, the possibilities to tune network services to the bidding of user applications are immense. Whereas network function virtualization aspires to virtualize diversified network hardware into effortlessly serviced software solutions provisioned on commodity servers, the thesis aims to research the implications of moving event processing application context on to virtualized network components. The hypothesis to begin with is that offloading of application context onto the underlying network allows network architects to tune their services to latency sensitive applications. It also provides better insights into the traffic characteristics of the application which may be used for designing better staged processing, load balancing and reducing the overall clog on the network I/O stack. Finally, it presents network operators with opportunities to explore further monetization of the network and virtualization infrastructure and pro-actively scrutinize revenue models which blur the line between application land and the network.
	\end{abstract}